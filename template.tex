\documentclass[9pt,a4paper]{jarticle}	%% 日本語用, 基本フォントサイズ9pt, A4用紙
%\documentclass[9pt,a4paper]{article}	%% For English
\usepackage{symposium}


%% 主題等を設定
\title{主題 Title(16pt)}	%% 主題設定
%\subtitle{副題(あれば) Subtitle(if necessary)(14pt)}	%% 副題設定(書かなくてもいい)
\author{著者 Author(12pt)}	%% 著者設定
\affiliation{所属 Affiliation(12pt)}	%% 所属設定


%% 概要をここに書く
\abstract{
研究のまとめ、概要について書いてください(目安:日本語で500文字以内、英数字で200語以内)。
}


\begin{document}
\maketitle	%% 主題、副題、著者、所属、概要を配置


\section{はじめに}
これはシンポジウムの予稿テンプレートです。テンプレートを使うときはsymposium.styというファイルをtexファイルと同じ場所に置いてください。
この資料をもとにシンポジウムで発表する内容をまとめた2〜6ページ程度の予稿を作成して下さい。

フォントサイズは原則${\rm 9pt}$とします。
セクションを始める場合は\textbackslash section\{セクション名\}のように書いてください。
脚注を書きたいときは\textbackslash footnote\{\}を使ってください\footnote{そうするとこのように表示されます。}。


\subsection{サブセクション}
サブセクションを始める場合は\textbackslash subsection\{サブセクション名\}のように書いてください。


\section{原稿作成の手引き}
このセクションでは図、表、数式を挿入する方法について書いておきます。


\subsection{図}
template\_tex.texの44〜51行目のようにfigure環境を使うと以下の図~\ref{fig:ACO}のように図が挿入できます。
なお本文中で図を参照するときは\textbackslash ref\{fig:ACO\}のようにキーワードを指定し、何度かコンパイルし直してください。
%
\begin{figure}[h]	%% 大括弧内は挿入位置指定
	\centering		%% センタリング
	\includegraphics
		[width=\columnwidth-20mm]	%% 幅指定
		{Picture1.png}	%% Picture1.pngを挿入
	\caption{ACOのグラフ}	%% キャプション
	\label{fig:ACO}		%% 本文で参照する際のキーワード
\end{figure}


\subsection{表}
表は書くのが面倒なのであらかじめWebアプリ\footnote{例えば \url{https://www.tablesgenerator.com/}}などでLatex用コードを出力してもらうといいでしょう。
template\_tex.texの58〜71行目のようにtable環境とtabular環境を使うと以下の表~\ref{tab:tenki}のようになります。
%
\begin{table}[h]
	\centering
	\caption{天気}
	\begin{tabular}{|c|c|c|}
		\hline
		日付  & 午前       & 午後     \\
		\hline
		9.1   & 晴れ       & 晴れ時々曇り \\
		9.2   & 曇りのち雨 & 雨      \\
		9.3   & 晴れ       & 晴れ     \\
		\hline
	\end{tabular}
	\label{tab:tenki}
\end{table}


\subsection{数式}
equation環境(単一の数式専用)やalign環境(複数でも大丈夫)が使用できます。
以下では78〜81行目のalign環境を用いた書き方による式~(\ref{eq:Pythagoras})を示します。
%
\begin{align}
a^2 + b^2 = c^2
\label{eq:Pythagoras}
\end{align}



\section{参考文献の書き方}
91〜95行目に示すように\textbackslash end\{document\} の上にthebibliography環境と\textbackslash bibitem コマンドを使って参考文献リストを記述し、
本文中の引用した箇所で\textbackslash cite コマンドを使ってそれを明示してください(こんな感じ~\cite{AritaSuzuki2000ALIFE})。



\setcounter{section}{\thesection+1}	%% sectionの書式をいじったためかこの1行が必要
\begin{thebibliography}{99}		%% 参考文献リスト用の環境
%% \bibitem{ラベル名} 論文情報 の順で書いていく
\bibitem{AritaSuzuki2000ALIFE} T. Arita, R. Suzuki. (2000), Interactions between Learning and Evolution: The Outstanding Strategy Generated by the Baldwin Effect, Proceedings of the Seventh Interernational Conference on Artificial Life, 2000, pp 196--205. 
\end{thebibliography}



\end{document}

